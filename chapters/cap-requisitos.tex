\chapter{\label{chap:requi}Requisitos}
Nesta seção, serão apresentados os requisitos funcionais e não funcionais levantados durante o processo de modelagem de software.

\section{Requisitos Funcionais}
\begin{itemize}
    \item O sistema deve permitir a inserção de dados sobre restaurantes;
    \item O sistema deve permitir que o usuário mantenha uma lista de locais favoritos;
    \item O sistema deve armazenar os itens do cardápio de acordo com a ontologia desenvolvida para restaurantes;
    \item O sistema deve informar, de forma atualizada, o número de instâncias dos itens apresentados no cardápio;
    \item O sistema deve permitir a edição de dados relacionados ao usuário e aos restaurantes, os quais foram previamente adicionados à aplicação;
    \item Através da ferramenta de localização, o sistema deve apresentar o cardápio ao usuário, caso o local esteja registrado no aplicativo;
    \item O usuário deve indicar quando alguma informação (previamente inserida no aplicativo) está desatualizada.
\end{itemize}

\section{Requisitos Não-Funcionais}
\begin{itemize}
    \item A aplicação deve ser desenvolvida na plataforma Android;
    \item A aplicação deve ser desenvolvida em Java, através da IDE Android Studio;
    \item A aplicação deve fazer uso das ferramentas de localização e TalkBack disponíveis em dispositivos Android;
    \item A aplicação deve utilizar um banco de dados SQLite para o armazenamento de dados;
    \item A aplicação deve seguir os princípios do W3C no quesito de acessibilidade para usuários com deficiência visual;
    \item A aplicação deve seguir os princípios de ergonomia relacionados à interação humano-computador, sugeridos por Cybis, Betiol e Faust (2015);
    \item A aplicação deve se guiar pelos padrões de design para aplicativos móveis.
\end{itemize}